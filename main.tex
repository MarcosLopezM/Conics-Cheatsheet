\documentclass{article}

\usepackage{amsfonts, amsmath,amssymb,amsthm}   % Paquetes de simbología matemática básica
\usepackage{lmodern,microtype,bm}     % Fuente y espaciado entre letras; lo deja bonito
\usepackage{dsfont, graphicx}
\usepackage{mathrsfs, halloweenmath,xcolor}
\usepackage{kantlipsum}
\usepackage{MnSymbol}
\usepackage{mathtools}
\usepackage{multicol,titlesec}
\usepackage[shortlabels]{enumitem}    % Continuar listas en mini-páginas distintas
\usepackage{physics}
\usepackage[english,spanish]{babel}   % Cambia los comandos de texto predeterminados (capítulos, 			                                        secciones, bibliografía, etc.) a español
\decimalpoint
\usepackage[style=mexican]{csquotes}  % Comillas y otros elementos de citación
\textwidth 16cm                       % Ancho
\oddsidemargin -0.0cm                 % Espacio de margen (como es formato de libro, los margenes se 		                                        declaran para páginas pares e impares
\usepackage[spanish]{todonotes}
\usepackage{xfrac}
\usepackage[unicode=true]{hyperref}

\newtheoremstyle{definicion}% name
{3pt}% Space above
{3pt}% Space below
{}% Body font
{}% Indent amount
{\color{blue}\bfseries}% Theorem head font
{.}% Punctuation after theorem head
{.5em}% Space after theorem head
{}%
\theoremstyle{definicion}
\newtheorem{definicion}{Def.}

\theoremstyle{definition}             % Con el paquete amsmath se pueden personalizar los estilos de
\newtheorem*{inst}{Instrucciones}

\theoremstyle{definition}             % Con el paquete amsmath se pueden personalizar los estilos de
\newtheorem{sol}{Solución}

\theoremstyle{definition}
\newtheorem{record}{Recordatorio}

\theoremstyle{definition}
\newtheorem{properties}{Propiedades}

\newtheoremstyle{observacion}% name
{3pt}% Space above
{3pt}% Space below
{}% Body font
{}% Indent amount
{\color{red}\bfseries}% Theorem head font
{.}% Punctuation after theorem head
{.5em}% Space after theorem head
{}%
\theoremstyle{observacion}
\newtheorem{obs}{Obs.}

\theoremstyle{definition}
\newtheorem{prop}{Proposición}

\theoremstyle{plain}
\newtheorem{lemma}{Lema}
\newtheorem{theorem}{Teorema}

\theoremstyle{definition}
\newtheorem{exe}{Ejemplo}

\newtheoremstyle{afirmacion}% name
{3pt}% Space above
{3pt}% Space below
{}% Body font
{}% Indent amount
{\color{green!40!black}\bfseries}% Theorem head font
{.}% Punctuation after theorem head
{.5em}% Space after theorem head
{}%
\theoremstyle{afirmacion}
\newtheorem{corollary}{Corolario}

\newtheoremstyle{notation}% name
{3pt}% Space above
{3pt}% Space below
{}% Body font
{}% Indent amount
{\color{magenta}\bfseries}% Theorem head font
{.}% Punctuation after theorem head
{.5em}% Space after theorem head
{}%
\theoremstyle{notation}
\newtheorem{notation}{Notación}

\theoremstyle{definition}
\newtheorem{eje}{Ejercicio}

\setlength{\parindent}{2em}           % Sangría
\setlength{\parskip}{0.5em}           % Espacio entre párrafos

\graphicspath{{./img}}
\DeclareMathOperator*{\ang}{\overrightarrow{\textrm{ang}}}

\title{\Huge{El plano euclidio}}
\author{Geometría Analítica I}
\date{\today}

\begin{document}
    \maketitle
    
    
    \section{Círculos}

    Ya hemos definido y usado extensamente el círculo unitario \(\mathbb{S}^{1}\), determinado por la ecuación

    \begin{equation*}
        x^{2} + y^{2} = 1.
    \end{equation*}

    Consideremos ahora cualquier otro círculo \(\mathcal{C}\). Tiene un \textcolor{blue}{centro}  \(\vb{p} = (h, k)\), un \textcolor{blue}{radio} \(r > 0\), y el el lugar geométrico de los puntos cuya distancia \(\vb{p}\) es \(r\). En otras palabras, \(\mathcal{C} = \left\lbrace \vb{x} \in \mathbb{R}^{2} \mid \textrm{d}(\vb{x}, \vb{p}) = r\right\rbrace\); o bien, \(\mathcal{C}\) está definido por la ecuación

    \begin{equation*}
        \textrm{d}(\vb{x}, \vb{p}) = r.
    \end{equation*}

    Puesto que ambos lados de esta ecuación son positivos, es equivalente a la igualdad de sus cuadrados que en coordenadas toma la forma

    \begin{equation}
        (x - h)^{2} + (y - k)^{2} = r^{2}.
    \end{equation}

    De tal manera que todos los círculos de \(\mathbb{R}^{2}\) están determinados por una ecuación cuadrática en las variables \(x\) y \(y\). Cuando la ecuación tiene la forma anterior, podemos ``leer'' toda la información geométrica (el centro y el radio). Pero en general viene disfrazada como

    \begin{equation*}
        x^{2} + y^{2} -2hx - 2ky = (r^{2} - h^{2} - k^{2}) 
    \end{equation*}

    que es, claramente, su expresión desarrollada.

    También podemos expresar la ecuación de un círculo de manera vectorial, pues el círculo \(\mathcal{C}\) con centro \(\vb{p}\) y radio \(r\) está claramente definido por la ecuación

    \begin{equation*}
        (\vb{x} - \vb{p}) \vdot (\vb{x} - \vb{p}) = r^{2}.
    \end{equation*}

    Esta ecuación, que llamaremos \textcolor{blue}{ecuación vectorial del círculo}, se puede también reescribir como

    \begin{equation*}
        \vb{x} \vdot \vb{x} - 2\vb{p} \vdot \vb{x} + \vb{p} \vdot \vb{p} = r^{2}.
    \end{equation*}

    \subsection{Tangentes y polares}

    Observemos primero que las líneas \textcolor{blue}{tangentes} a un círculo son las normales a los \textcolor{blue}{radios}. Efectivamente, si \(\vb{a}\) es un punto del círculo \(\mathcal{C}\) dado por la ecuación vectorial

    \begin{equation*}
        (\vb{x} - \vb{p}) \vdot (\vb{x} - \vb{p}) = r^{2},
    \end{equation*}

    entonces su tangente es la recta \(\ell\) normal a \((\vb{a} - \vb{p})\) y que pasa por \(\vb{a}\). Pues \(\vb{a}\) es el punto más cercano a \(\vb{p}\) en esta recta, de tal manera que para cualquier otro punto \(\vb{x} \in \ell\) se tiene que \(\textrm{d}(\vb{x}, \vb{p}) > r\). Como el círculo \(\mathcal{C}\) parte el plano en dos pedazos, entonces \(\ell\) está contenida en el exterior salvo por el punto \(\vb{a} \in \mathcal{C}\); esta será nuestra definición de \textcolor{blue}{tangente}.

    Claramente, la recta \(\ell\) está dada por la ecuación

    \begin{equation*}
        \vb{x} \vdot (\vb{a} - \vb{p}) = \vb{a} \vdot (\vb{a} - \vb{p}).
    \end{equation*}

    Restando \(\vb{p} \vdot (\vb{a} - \vb{p})\) a ambos lados se obtiene

    \begin{align*}
        \vb{x} \vdot (\vb{a} - \vb{p}) - \vb{p} \vdot (\vb{a} - \vb{p}) &= \vb{a} \vdot (\vb{a} - \vb{p}) - \vb{p} \vdot (\vb{a} - \vb{p}),\\
        (\vb{x} - \vb{p}) \vdot (\vb{a} - \vb{p}) &= (\vb{a} - \vb{p}) \vdot (\vb{a} - \vb{p}),\\
        (\vb{x} - \vb{p}) \vdot (\vb{a} - \vb{p}) &= r^{2},
    \end{align*}

    pues \(\vb{a} \in \mathcal{C}\). La ecuación de la tangente se obtiene entonces al sustituir el punto en una de las dos apariciones de la variable \(\vb{x}\) en la ecuación vectorial.

    Para cualquier otro punto en el plano, \(\vb{a}\) digamos, que nos sea el centro (\(\vb{a} \neq \vb{p}\)), el mismo proceso algebraico nos da la ecuación de una recta, llamada \textcolor{blue}{la polar} de \(\vb{a}\) respecto al círculo \(\mathcal{C}\), y que denotaremos con \(\ell_{\vb{a}}\); es decir, sea

    \begin{equation*}
        \ell_{\vb{a}} \colon\quad (\vb{x} - \vb{p}) \vdot (\vb{a} - \vb{p}) = r^{2}.
    \end{equation*}

    Ya hemos visto que cuando \(\vb{a} \in \mathcal{C}\), su polar \(\ell_{\vb{a}}\) es su tangente. Ahora veremos que si \(\vb{a}\) está en el interior del círculo, entonces \(\ell_{\vb{a}}\) no se intersecta con él, y que si está en el exterior, entonces lo corta, y además lo corta en los dos puntos de \(\mathcal{C}\) a los cuales se pueden trazar tangentes.

    \missingfigure[]{Insertar diagrama correspondiente a la recta polar que se encuentra en la página 87}

    Para esto, expresamos la ecuación de \(\ell_{\vb{a}}\) en su forma normal, 

    \begin{equation*}
        \vb{x} \vdot (\vb{a} - \vb{p}) = r^{2} + \vb{p} \vdot (\vb{a} - \vb{p}).
    \end{equation*}

    Esto indica que \(\ell_{\vb{a}}\) es perpendicular al vector qeu va de \(\vb{p}\) a \(\vb{p}\).

    Ahora veamos cuál es el punto de intersección con la recta que pasa por \(\vb{p}\) y \(\vb{a}\). Parametricemos esta última recta con \(\vb{p}\) de cero y \(\vb{a}\) de uno y podemos despejar \(t\) al sustituir en la variable \(\vb{x}\) de la ecuación anterior, para obtener 

    \begin{equation*}
        t = \dfrac{r^{2}}{(\vb{a} - \vb{p}) \vdot (\vb{a} - \vb{p})} = \dfrac{r^{2}}{\textrm{d}(\vb{p}, \vb{a})^{2}}.
    \end{equation*}

    Entonces la distancia de \(\vb{p}\) a \(\ell_{\vb{a}}\) es 

    \begin{equation*}
        \textrm{d}(\vb{p}, \ell_{\vb{a}}) = t\textrm{d}(\vb{p}, \vb{a}) = \dfrac{r^{2}}{\textrm{d}(\vb{p}, \vb{a})} = \left(\dfrac{r}{\textrm{d}(\vb{p}, \vb{a})}\right)r
    \end{equation*}

    y tenemos lo primero que queríamos: si \(\textrm{d}(\vb{p}, \vb{a}) < r\) entonces \(\textrm{d}(\vb{p}, \ell_{\vb{a}}) > r\); y al revés, si \(\textrm{d}(\vb{p}, \vb{a}) > r\) entonces \(\textrm{d}(\vb{p}, \ell_{\vb{a}}) < r\). Si el punto \(\vb{a}\) está muy cerca de \(\vb{p}\), su polar está muy lejos, y al revés, sus distancias al centro \(\vb{p}\) se comportan como inversos pero ``alrededor de \(r\)''.

    Supongamos ahora que \(\textrm{d}(\vb{p}, \vb{a}) > r\), y sea \(\vb{c}\) un punto en \(\ell_{\vb{a}} \cap \mathcal{C}\). Puesto que \(\vb{c} \in \ell_{\vb{a}}\), se cumple la ecuación 

    \begin{equation*}
        (\vb{c} - \vb{p}) \vdot (\vb{a} - \vb{p}) = r^{2}.
    \end{equation*}

    Pero entonces \(\vb{a}\) cumple la ecuación \(\ell_{\vb{c}}\) que es la tangente a \(\mathcal{C}\) en \(\vb{c}\); es decir, la línea de \(\vb{a}\) a \(\vb{c}\) es tangente al círculo.

    Nótese que el argumento anterior es mucho más general: demuestra que para cualesquiera dos puntos \(\vb{a}\) y \(\vb{b}\) se tiene que 

    \begin{equation*}
        \vb{a} \in \ell_{\vb{b}}\quad \Leftrightarrow\quad \vb{b} \in \ell_{\vb{a}}.
    \end{equation*}

    Y los puntos del círculo son los únicos para los cuales se cumple \(\vb{a} \in \ell_{\vb{a}}\).

    \section{Elipses}

    Las \textcolor{blue}{elipses} son el lugar geométrico de los puntos cuya suma de distancias a dos fijos llamados \textcolor{blue}{focos}  es constante. De tal manera que una elipse \(\mathcal{E}\) queda determinada por la ecuación 

    \begin{equation*}
        \textrm{d}(\vb{x}, \vb{p}) + \textrm{d}(\vb{x}, \vb{q}) = 2a,
    \end{equation*}

    donde \(\vb{p}\) y \(\vb{q}\) son los focos y \(a\) es una constaste positiva, llamada \textcolor{blue}{semieje mayor}, tal que \(2a > \textrm{d}(\vb{p}, \vb{q})\).

    Incluimos el coeficiente \(2\) en la constante para que quede claro que si los focos coinciden, \(\vb{p} = \vb{q}\); entonces se obtiene un círculo de radio \(a\) y centro en el foco; así que los círculos son elipses muy especiales.

    Supongamos que el \textcolor{blue}{centro} de la elipse \(\mathcal{E}\), \emph{i.e.}, el punto medio entre los focos, está en el origen y que además los focos están en el eje \(x\). Entonces tenemos que \(\vb{p} = (c, 0)\) y \(\vb{q} = (-c, 0)\), para alguna \(c\) tal que \(0 < c < a\) (donde ya suponemos que la elipse no es círculo al pedir \(0 < c\)). Es fácil ver entonces que la intersección de \(\mathcal{E}\) con el eje \(x\) consiste en los puntos \((a, 0)\) y \((-a, 0)\), pues la ecuación anterior para puntos \((x, 0)\) es 

    \begin{equation*}
        \abs{\vb{x} - \vb{c}} + \abs{\vb{x} + \vb{c}} = 2a,
    \end{equation*}

    que solo tiene las soluciones \(x = a\) y \(x = -a\), y de ahí el ombre de \textcolor{blue}{semieje mayor} para la constante \(a\). Como el eje \(y\) es ahora la mediatriz de los focos, en él, es decir en los puntos \((0, y)\), la primera ecuación de la ellipse se vuelve
    
    \begin{equation*}
        \sqrt{c^{2} + y^{2}} = a,
    \end{equation*}

    que tiene soluciones \(y = \pm b\), donde \(b > 0\). Se le llama el \textcolor{blue}{semieje menor} de la elipse \(\mathcal{E}\), y es tal que

    \begin{equation*}
        b^{2} = a^{2} - c^{2}. 
    \end{equation*}

    Ahora sí, consideramos la primera ecuación de la elipse, que con \(\vb{x} = (x, y)\) y la definición de nuestros focos, se expresa

    \begin{equation*}
        \sqrt{(x - c)^{2} + y^{2}} + \sqrt{(x - c)^{2} + y^{2}}= 2a.
    \end{equation*}

    Pasamos una de las dos raíces al otro lado y elevamos al cuadrado, 

    \begin{equation*}
        a\sqrt{(x + c)^{2} + y^{2}} = a^{2} + cx.	
    \end{equation*}

    Elevando de nuevo al cuadrado, nos deshacemos de la raíz, y después, al agrupar términos, obtenemos

    \begin{equation*}
        b^{2}x^{2} + a^{2}y^{2} = a^{2}b^{2},
    \end{equation*}

    que, al dividir entre \(a^{2}b^{2}\), se escribe finalmente como

    \begin{equation*}
        \dfrac{x^{2}}{a^{2}} + \dfrac{y^{2}}{b^{2}} = 1,
    \end{equation*}

    que es la llamada \textcolor{blue}{ecuación canónica de la elipse}.

    \section{Hipérbolas}

    La \textcolor{blue}{hipérbola} está definida como el lugar geométrico de los punts cuya diferencia (en valor absoluto) de sus distancias a dos puntos fijos \(\vb{p}\) y \(\vb{q}\), llamados focos, es constante. Entonces, una hipérbola \(\mathcal{H}\) está definida por la ecuación

    \begin{equation*}
        \abs{\textrm{d}(\vb{x}, \vb{p}) - \textrm{d}(\vb{x}, \vb{q})} = 2a,
    \end{equation*}

    donde \(a > 0\), y además \(2a < \textrm{d}(\vb{p}, \vb{q}) \eqqcolon 2c\). Si tomamos como focos a \(\vb{p} = (c, 0)\) y \(\vb{q} = (-c, 0)\), esta ecuación toma la forma

    \begin{equation*}
        \abs{\sqrt{(x - c)^{2} + y^{2}} - \sqrt{(x + c)^{2} + y^{2}}} = 2a.
    \end{equation*}

    Veremos a continuación que es equivalente a 

    \begin{equation*}
        \dfrac{x^{2}}{a^{2}} - \dfrac{y^{2}}{b^{2}} = 1,
    \end{equation*}

    donde \(b > 0\) está definida por \(a^{2} + b^{2} = c^{2}\). A esta última ecuación se le llama la \textcolor{blue}{ecuación canónica de la hipérbola}.	

    Como las dos primera ecuaciones involucran valores absolutos, entonces tienen dos posibilidades que corresponden a las dos ramas de la hipérbola. En una de ellas la distancia a uno de los focos es mayor y en la otra se invierten los papeles. Se tienen dos posibilidades:

    \begin{align*}
        \sqrt{(x - c)^{2} + y^{2}} &= 2a + \sqrt{(x + c)^{2} + y^{2}} \\
        \sqrt{(x + c)^{2} + y^{2}} &= 2a + \sqrt{(x - c)^{2} + y^{2}};
    \end{align*}

    la primera corresponde a la rama donde \(x < 0\) y luego volviendo a elevar al cuadrado y simplificando, se obtiene, de cualquiera de las dos ecuaciones anteriores, la ecuación 

    \begin{equation*}
        (a^{2} - c^{2})x^{2} + a^{2}y^{2} = a^{2}(a^{2} - c^{2}).
    \end{equation*}

    Y de aquí, al sustituir \(-b^{2} = a^{2} - c^{2}\), y dividir entre \(-a^{2}b^{2}\) se obtiene la ecuación canónica.

    \section{Parábolas}

    Una \textcolor{blue}{parábola} es el lugar geométrico de los puntos que equidistan de un punto \(\vb{p}\) (llamado su \textcolor{blue}{foco}) y una recta \(\ell\), llamada su \textcolor{blue}{directriz}, donde \(\vb{p} \in \ell\). Es decir, está definida por la ecuación

    \begin{equation*}
        \textrm{d}(\vb{x}, \vb{p}) = \textrm{d}(\vb{x}, \ell).
    \end{equation*}

    Tomemos un ejemplo sencillo con el foco en el eje \(y\), la directriz paralela al eje \(x\), y que además pase por el origen. Tenemos entonces \(\vb{p} = (0, c)\), donde \(c > 0\) digamos, y \(\ell \colon y = -c\); de tal manera que la parábola queda determinada por la ecuación 

    \begin{equation*}
        \sqrt{x^{2} + (y - c)^{2}} = \abs{y + c}.
    \end{equation*}

    Como ambos lados de la ecuación son positivos, esta es equivalente a la igualdad de sus cuadrados que da 

    \begin{align*}
        x^{2} + y^{2} -2cy + c^{2} &= y^{2} + 2cy + c^{2}, \\
        x^{2} &= 4cy.        
    \end{align*}

    De tal manera que la gráfica de la función \(x^{2}\) (es decir, \(y = x^{2}\)) es una parábola con foco \((0, \frac{1}{4})\) y directriz \(y = -\frac{1}{4}\). Veremos ahora que cualquier parábola cumple ``la propiedad de ser gráfica de una función'' respecto de su directriz.

    Sea \(\mathcal{P}\) la parábola con foco \(\vb{p}\) y directriz \(\ell\) (donde \(\vb{p} \in \ell\)). Dado un punto \(y_{0} \in \ell\), está claro que los puntos del plano cuyo distancia a \(\ell\) coincide con su distancia a \(y_{0}\) son precisamente los de la normal a \(\ell\) que pasa por \(y_{0}\), que llamaremos \(v_{0}\). Por otro lado, la mediatriz entre \(y_{0}\) y \(\vb{p}\), llamémosla \(\eta_{0}\), consta de los puntos cuyas distancias a \(\vb{p}\) y a \(y_{0}\) coinciden. Por lo tanto la intersección de \(\eta_{0}\) y \(v_{0}\) está en la parábola \(\mathcal{P}\), es decir, \(x_{0} = v_{0} \cap \eta_{0} \in \mathcal{P}\), pues

    \begin{equation*}
        \textrm{d}(\vb{x}_{0}, \ell) = \textrm{d}(\vb{x}_{0}, \vb{y}_{0}) = \textrm{d}(\vb{x}_{0}, \vb{p}).
    \end{equation*}

    Pero además \(\vb{x}_{0}\) es el único punto en la normal \(v_{0}\) que está en \(\mathcal{P}\). esta es ``la propiedad de la gráfica'' a la que nos referíamos. 

    Podemos concluir aún más: que la mediatriz \(\eta_{0}\) es la tangente a \(\mathcal{P}\) en \(\vb{x}_{0}\). Pues para cualquier otro punto \(\vb{x} \in \eta_{0}\) se tiene que su distancia a \(\ell\) es menor que su distancia a \(\vb{y}_{0}\) que es su distancia a \(\vb{p}\) (\(\textrm{d}(\vb{x}_{0}, \ell) < \textrm{d}(\vb{x}, \vb{y}_{0}) = \textrm{d}(\vb{x}, \vb{p})\)), y entonces \(\vb{x} \notin \mathcal{P}\). De hecho, la parábola \(\mathcal{P}\) parte el plano en dos pedazos, los puntos más cerca de \(\vb{p}\) que de \(\ell\) (definidos por la desigualdad \(\textrm{d}(\vb{x}, \vb{p}) \leq \textrm{d}(\vb{x}, \ell)\)) y los que están más cerca de \(\ell\) que de \(\vb{p}\) (dados por \(\textrm{d}(\vb{x}, \ell) \leq \textrm{d}(\vb{x}, \vb{p})\) en donde está \(\eta_{0}\)), que comparten la frontera donde estas distancias coinciden. Así que \(\eta_{0}\) pasa \textcolor{blue}{tangente} a \(\mathcal{P}\) en \(\vb{0}\), pues \(\vb{x}_{0} \in \eta_{0} \cap \mathcal{P}\) y además \(\eta_{0}\) se queda en el lado de \(\mathcal{P}\).

    \missingfigure[]{Insertar diagrama correspondiente a la explicación anterior que es encuentra en la página 93}
\end{document}